\documentclass{article}
\usepackage[utf8]{inputenc}

\usepackage[utf8]{inputenc}
\usepackage[greek, english]{babel}
\usepackage{amsmath,amsfonts,amssymb}
\usepackage{hhline}
\usepackage{alphabeta}
\usepackage{lmodern}
\usepackage[margin = 2.5cm, paperwidth=8.5in, paperheight=11in]{geometry}
\usepackage{graphicx, wrapfig}
\usepackage{minted}


\begin{document}
\begin{wrapfigure}[5]{l}{0.2\textwidth}
\includegraphics[scale=0.11]{pyrforos.png}
\end{wrapfigure} 
\noindent
Εθνικό Μετσόβιο Πολυτεχνείο\\
Σχολή Ηλεκτρολόγων Μηχανικών και Μηχανικών Υπολογιστών\\
Τομέας Τεχνολογίας Πληροφορικής και Υπολογιστών\\
\textbf{Εργαστήριο Μικροϋπολογιστών}\\
Χειμερινό Εξάμηνο 2021-2022\\
\vspace{2.5cm}
\vspace{-5em}
\hrule \vspace{.2em}
\hrule
\vspace{1cm}
\centering
{\Large \textbf{5η Εργαστηριακή Άσκηση}\\}
\vspace{0.2cm}
\raggedright
$~~~~$\\
Μέλος 1: Λαγός Αναστάσιος - 03113531\\
Μέλος 2: Αντώνιος Δημήτριος Αλικάρης - 03118062 \\

\section*{Έλεγχος έντασης LEDs με χρήση PWM κυματομορφής}

\subsection*{Γενική ιδέα}
Για να αυξομειώσουμε την ένταση των LEDs, χρησιμοποιούμε τον χρονιστή/μετρητή TIMER/COUNTER0 για να παράξουμε μία PWM κυματομορφή συχνότητας 4kHz της οποίας το duty cycle μεταβάλλεται από το πληκτρολόγιο(με το πλήκτρο "1" αυξάνεται, ενώ με το πλήκτρο "2" μειώνεται). Λόγω απομακρυσμένης πρόσβασης στην πλακέτα, για εξακρίβωση επιτυχής λειτουργίας, συνδέουμε την κυματομορφή ως είσοδο στον μετατροπέα ADC και εμφανίζουμε την τάση στην οθόνη LCD.\\
Ο χρονιστής/μετρητής TIMER/COUTER1 χρησιμοποιείται για την λειτουργία του ADC κάθε 100ms.\\
Το avr project περιλαμβάνει και 2 αρχεία .S που περιέχουν τις δύο απαραίτητες ρουτίνες για την λειτουργία της LCD(LCD\_init και LCD\_show) γραμμένες σε assembly.

\subsection*{Κύριος Κώδικας σε c}

\begin{minted}[linenos,frame=single,breaklines=true]{c}

#undef F_CPU
#define F_CPU 8000000UL

#ifndef __DELAY_BACKWARD_COMPATIBLE__
#define __DELAY_BACKWARD_COMPATIBLE__
#endif

#include <avr/io.h>
#include <util/delay.h>
#include <avr/interrupt.h>
#include <stdlib.h>
#include <string.h>

#define NOP(){__asm__ __volatile__("nop");} //assembly nop

/*Keypad Functions*/
unsigned int scan_row_sim(unsigned int row);
void scan_keypad_sim();
unsigned int scan_keypad_rising_edge_sim(unsigned int flick_time);
unsigned char keypad_to_ascii_sim();

unsigned int swap(unsigned int val);

/*Delay functions*/
void wait_usec(unsigned int delay);
void wait_msec(unsigned int delay);

/*LCD Assembly*/
//External LCD routines in assembly. Source code shown below
extern void LCD_init();
extern void LCD_show(unsigned char cha);

/*Globals*/
unsigned int buttons[2], ram[2];
volatile unsigned int adc_output = 0;

/*function that delays for delay time*/
void wait_usec(unsigned int delay){
	unsigned int i;
	for(i = 0; i < (delay/10); i++) {    //10 usec delay for delay/10 times
		_delay_us(10);
	}
	if (delay % 10) {    //delay for the remainder accordingly
		_delay_us(delay % 10);
	}
}
/*same function as wait_usec but for milliseconds*/
void wait_msec(unsigned int delay) {
	unsigned int i ;
	for(i = 0; i < (delay / 10); i++){
		_delay_ms(10);
	}
	if(delay % 10) {
		_delay_ms(delay % 10);
	}
}
/*Function that does __asm__("swap")*/
unsigned int swap(unsigned int val) {
	return ((val & 0x0F) << 4 | (val & 0xF0) >> 4);
}

/*Function that returns which columns are pressed for a given row*/
unsigned int scan_row_sim(unsigned int row) {
	PORTC = row;//Search in line row.
	wait_usec(500); //Delay required for a successful remote operation
	
	NOP();
	NOP();//Delay to allow for a change of state
	
	return PINC & 0x0F; //Return 4 LSB
}
/*Function that scans the whole keypad*/
void scan_keypad_sim() {
	buttons[1] = swap(scan_row_sim(0x10));// A 3 2 1
	buttons[1] += scan_row_sim(0x20);// A 3 2 1 B 6 5 4

	buttons[0] = swap(scan_row_sim(0x40));// C 9 8 7
	buttons[0] += (scan_row_sim(0x80)); // C 9 8 7 D # 0 *

	PORTC = 0x00; // added only for the remote operation
	return;
}
/*Function that checks which buttons where pressed since its last call*/
unsigned int scan_keypad_rising_edge_sim(unsigned int flick_time) {
	scan_keypad_sim(); // do the first scan
	unsigned int temp[2]; // store first scan
	temp[0] = buttons[0];
	temp[1] = buttons[1];
	wait_msec(flick_time); //wait for flick time
	
	scan_keypad_sim();  //scan second time
	buttons[0] &= temp[0]; //remove flick values
	buttons[1] &= temp[1];

	temp[0] = ram[0];   //get the last state from previous call to rising_edge from "RAM"
	temp[1] = ram[1];
	ram[0] = buttons[0];    //update the new previous state
	ram[1] = buttons[1];
	buttons[0] &= ~temp[0]; //Keep values that change from 1 to 0
	buttons[1] &= ~temp[1];
	
	return (buttons[0] || buttons[1]);
}
/*Function that returns the ASCII code of button pressed*/
unsigned char keypad_to_ascii_sim() {
	unsigned int select;
	for(select = 0x01; select <= 0x80; select <<= 1) {
		switch(buttons[0] & select) {
			case 0x01:
			return '*';
			case 0x02:
			return '0';
			case 0x04:
			return '#';
			case 0x08:
			return 'D';
			case 0x10:
			return '7';
			case 0x20:
			return '8';
			case 0x40:
			return '9';
			case 0x80:
			return 'C';
		}
	}
	for(select = 0x01; select <= 0x80; select <<= 1) {
		switch(buttons[1] & select) {
			case 0x01:
			return '4';
			case 0x02:
			return '5';
			case 0x04:
			return '6';
			case 0x08:
			return 'B';
			case 0x10:
			return '1';
			case 0x20:
			return '2';
			case 0x40:
			return '3';
			case 0x80:
			return 'A';
		}
	}
	return 0;
}

void PWM_init() {
	//set TMR0 in fast PWM mode with non-inverted output, prescale=8
	TCCR0 = (1<<WGM00) | (1<<WGM01) | (1<<COM01) | (1<<CS01);
	DDRB |= (1<<PB3);
}

void ADC_init(){
	ADMUX = (1<<REFS0);
	ADCSRA = (1<<ADEN)|(1<<ADIE)|(1<<ADPS2)|(1<<ADPS1)|(1<<ADPS0);
}

ISR(TIMER1_OVF_vect){
	TCNT1 = 64755; //100ms
	ADCSRA |= (1<<ADSC);//adc start conversion enable
}

ISR(ADC_vect){
	adc_output = ADC;
}

//function which takes the ADC and shows in the lcd the converted voltage value
void convert_to_voltage_and_show(char adc_decimal_buffer[]) {
	float analog_input = (float) 5 * adc_output / 1024;
	
	char integer_part = (int) analog_input + '0';
	unsigned int decimal_part = (int) (analog_input * 100) % 100;
	itoa(decimal_part, adc_decimal_buffer, 10);
	
	LCD_show(integer_part);
	LCD_show('.');
	LCD_show(adc_decimal_buffer[0]);
	LCD_show(adc_decimal_buffer[1]);
	
	return;
}


int main(void)
{
	DDRD = 0xFF;	//Initialize LCD as output
	DDRC = 0xF0;	//Initialize PORTC and LEDs. Internal resistor pull up must be deactivated
	unsigned int duty = 0;
	char adc_decimal[2];
	unsigned int old_output = -1;
	ADC_init();
	LCD_init();
	PWM_init();
	TIMSK = (1<<TOIE1);
	TCCR1B = (1<<CS12)|(0<<CS11)|(1<<CS10);
	TCNT1 = 64755; //100ms
	sei();
	while(1) {
		unsigned char button_pressed;
		ram[0] = 0, ram[1] = 0;
		while(1) {
			if(scan_keypad_rising_edge_sim(15)) {
				button_pressed = keypad_to_ascii_sim();
				break;
			}
			if (old_output != adc_output) { //if the adc hasn't changed, dont show and compute the output again
				old_output = adc_output;
				LCD_init();
				LCD_show('V');
				LCD_show('o');
				LCD_show('1');
				LCD_show('\n');
				convert_to_voltage_and_show(adc_decimal);
			}
		}
		if ((button_pressed == '1') && duty < 255) {
			duty++;//increasing the upper pulse
			OCR0 = duty;
		}
		else if((button_pressed == '2') && duty > 0){
			duty--;//decreasing the upper pulse
			OCR0 = duty;
		}
	}
}


\end{minted}



\subsection*{Πρόσθετες συναρτήσεις σε assembly}

\subsection*{LCD\_init.S}

\begin{minted}[linenos,frame=single,breaklines=true]{nasm}

#define _SFR_ASM_COMPAT 1
#define __SFR_OFFSET 0

#include <avr/io.h>

.global LCD_init

LCD_init:
	rcall lcd_init_sim
	ret

	wait_msec:
		push r24
		push r25
		ldi r24, lo8(1000)
		ldi r25, hi8(1000)
		rcall wait_usec
		pop r25
		pop r24
		sbiw r24, 1
		brne wait_msec

		ret

	wait_usec:
		sbiw r24, 1 //2 cycles
		nop
		nop
		nop
		nop
		brne wait_usec //1 cycle the majority of the time

		ret

	write_2_nibbles_sim:
        push r24
        push r25
        ldi r24 ,lo8(6000) 
        ldi r25 ,hi8(6000)
        rcall wait_usec
        pop r25
        pop r24 
        push r24 
        in r25, PIND 
        andi r25, 0x0f 
        andi r24, 0xf0 
        add r24, r25 
        out PORTD, r24 
        sbi PORTD, PD3
        cbi PORTD, PD3
        push r24
        push r25
        ldi r24 ,lo8(6000)
        ldi r25 ,hi8(6000)
        rcall wait_usec
        pop r25
        pop r24
        pop r24
        swap r24
        andi r24 ,0xf0
        add r24, r25
        out PORTD, r24
        sbi PORTD, PD3
        cbi PORTD, PD3
        ret

	lcd_data_sim:
        push r24
        push r25
        sbi PORTD,PD2
        rcall write_2_nibbles_sim
        ldi r24,43
        ldi r25,0
        rcall wait_usec
        pop r25
        pop r24
        ret

	lcd_command_sim:
        push r24
        push r25
        cbi PORTD, PD2
        rcall write_2_nibbles_sim
        ldi r24, 39
        ldi r25, 0
        rcall wait_usec
        pop r25
        pop r24
        ret 

	lcd_init_sim:
        push r24
        push r25
        ldi r24, 40
        ldi r25, 0
        rcall wait_msec
        ldi r24, 0x30
        out PORTD, r24
        sbi PORTD, PD3
        cbi PORTD, PD3
        ldi r24, 39
        ldi r25, 0
        rcall wait_usec
        push r24
        push r25
        ldi r24,lo8(1000)
        ldi r25,hi8(1000)
        rcall wait_usec
        pop r25
        pop r24
        ldi r24, 0x30
        out PORTD, r24
        sbi PORTD, PD3
        cbi PORTD, PD3
        ldi r24,39
        ldi r25,0
        rcall wait_usec 
        push r24
        push r25
        ldi r24 ,lo8(1000)
        ldi r25 ,hi8(1000)
        rcall wait_usec
        pop r25
        pop r24
        ldi r24,0x20
        out PORTD, r24
        sbi PORTD, PD3
        cbi PORTD, PD3
        ldi r24,39
        ldi r25,0
        rcall wait_usec
        push r24
        push r25
        ldi r24 ,lo8(1000)
        ldi r25 ,hi8(1000)
        rcall wait_usec
        pop r25
        pop r24
        ldi r24,0x28
        rcall lcd_command_sim
        ldi r24,0x0c
        rcall lcd_command_sim
        ldi r24,0x01
        rcall lcd_command_sim
        ldi r24, lo8(1530)
        ldi r25, hi8(1530)
        rcall wait_usec
        ldi r24 ,0x06
        rcall lcd_command_sim
        pop r25
        pop r24
        ret

\end{minted}

\subsection*{LCD\_show.S}


\begin{minted}[linenos,frame=single,breaklines=true]{nasm}

#define _SFR_ASM_COMPAT 1
#define __SFR_OFFSET 0

#include <avr/io.h>

.global LCD_show

LCD_show:
	rcall lcd_data_sim
	ret

	wait_msec:
		push r24
		push r25
		ldi r24, lo8(1000)
		ldi r25, hi8(1000)
		rcall wait_usec
		pop r25
		pop r24
		sbiw r24, 1
		brne wait_msec

		ret

	wait_usec:
		sbiw r24, 1 //2 cycles
		nop
		nop
		nop
		nop
		brne wait_usec //1 cycle the majority of the time

		ret

	write_2_nibbles_sim:
        push r24
        push r25
        ldi r24 ,lo8(6000) 
        ldi r25 ,hi8(6000)
        rcall wait_usec
        pop r25
        pop r24 
        push r24 
        in r25, PIND 
        andi r25, 0x0f 
        andi r24, 0xf0 
        add r24, r25 
        out PORTD, r24 
        sbi PORTD, PD3
        cbi PORTD, PD3
        push r24
        push r25
        ldi r24 ,lo8(6000)
        ldi r25 ,hi8(6000)
        rcall wait_usec
        pop r25
        pop r24
        pop r24
        swap r24
        andi r24 ,0xf0
        add r24, r25
        out PORTD, r24
        sbi PORTD, PD3
        cbi PORTD, PD3
        ret

	lcd_data_sim:
        push r24
        push r25
        sbi PORTD,PD2
        rcall write_2_nibbles_sim
        ldi r24,43
        ldi r25,0
        rcall wait_usec
        pop r25
        pop r24
        ret

	lcd_command_sim:
        push r24
        push r25
        cbi PORTD, PD2
        rcall write_2_nibbles_sim
        ldi r24, 39
        ldi r25, 0
        rcall wait_usec
        pop r25
        pop r24
        ret 

	lcd_init_sim:
        push r24
        push r25
        ldi r24, 40
        ldi r25, 0
        rcall wait_msec
        ldi r24, 0x30
        out PORTD, r24
        sbi PORTD, PD3
        cbi PORTD, PD3
        ldi r24, 39
        ldi r25, 0
        rcall wait_usec
        push r24
        push r25
        ldi r24,lo8(1000)
        ldi r25,hi8(1000)
        rcall wait_usec
        pop r25
        pop r24
        ldi r24, 0x30
        out PORTD, r24
        sbi PORTD, PD3
        cbi PORTD, PD3
        ldi r24,39
        ldi r25,0
        rcall wait_usec 
        push r24
        push r25
        ldi r24 ,lo8(1000)
        ldi r25 ,hi8(1000)
        rcall wait_usec
        pop r25
        pop r24
        ldi r24,0x20
        out PORTD, r24
        sbi PORTD, PD3
        cbi PORTD, PD3
        ldi r24,39
        ldi r25,0
        rcall wait_usec
        push r24
        push r25
        ldi r24 ,lo8(1000)
        ldi r25 ,hi8(1000)
        rcall wait_usec
        pop r25
        pop r24
        ldi r24,0x28
        rcall lcd_command_sim
        ldi r24,0x0c
        rcall lcd_command_sim
        ldi r24,0x01
        rcall lcd_command_sim
        ldi r24, lo8(1530)
        ldi r25, hi8(1530)
        rcall wait_usec
        ldi r24 ,0x06
        rcall lcd_command_sim
        pop r25
        pop r24
        ret

\end{minted}
\end{document}
