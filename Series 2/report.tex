\documentclass[11pt]{article}

\usepackage[utf8]{inputenc}
\usepackage[greek, english]{babel}
\usepackage{alphabeta}
\usepackage{lmodern}
\usepackage[margin = 2.5cm, paperwidth=8.5in, paperheight=11in]{geometry}
\usepackage{diagbox}
\usepackage{caption}
\captionsetup[listing]{position=top}
\usepackage[newfloat]{minted}
\usepackage[normalem]{ulem}
\usepackage{listings}

\begin{document}

\begin{titlepage}
	\begin{center}
		{ 
			ΕΘΝΙΚΟ ΜΕΤΣΟΒΙΟ ΠΟΛΥΤΕΧΝΕΙΟ\\
			ΣΧΟΛΗ ΗΛΕΚΤΡΟΛΟΓΩΝ ΜΗΧΑΝΙΚΩΝ ΚΑΙ ΜΗΧΑΝΙΚΩΝ ΥΠΟΛΟΓΙΣΤΩΝ
		}
		
		\vfill

		{\Large
			Εργαστήριο Μικροϋπολογιστών
		}
		
		{\large
			$1^η$ εργαστηριακή άσκηση \\ mLab 8085
		}

		\vfill

		{
			Αναστάσιος Λαγός - 03113531\\
			Αντώνιος Δημήτριος Αλικάρης - 03118062
		}
	\end{center}
\end{titlepage}


\section*{Άσκηση 1}

\subsection*{Γενική Ιδέα}

Αφού διαβάσουμε τα 4 LSB της θύρας C και αποθηκεύοντας το καθένα σε έναν ξεχωριστό καταχωρητή, τα μεταφέρουμε όλα στο LSB και εκτελούμε διαδοχικά τις αντίστοιχες λογικές πράξεις. Τέλος εμφανίζουμε το αποτέλεσμα στην έξοδο.

\subsection*{Κώδικας σε assembly}

\begin{minted}[linenos,frame=single,breaklines=true]{nasm}

.org 0x0
rjmp reset

reset:
	; set r24 to 0x00
	clr r24
	; set PORTC as input
	out DDRC , r24
	; set r24 to 0xFF
	ser r24
	;set PORTB as output
	out DDRB , r24

main:
	;r16 = F0
	;r17 = F1
	;r18 = A
	;r19 = B
	;r20 = C
	;r21 = D

	in r18 , PINC	 ;get the input
	andi r18, 0x01	;get PC0

	in r19, PINC	    ;get the input
	andi r19, 0x02	 ;get PC1
	lsr r19		;shift right

	in r20, PINC
	andi r20, 0x04
	lsr r20
	lsr r20

	in r21, PINC
	andi r21, 0x08
	lsr r21
	lsr r21
	lsr r21

	; F0 = (A'B + B'CD)'

	mov r24, r18 ; r24 <- A
	com r24 ; r24 <- A'
	and r24, r19 ; r24 <- A'B
	mov r16, r24 ; F0 <- A'B

	mov r24, r19 ; r24 <- B
	com r24		;r24 <- B'
	and r24, r20	; r24 <- B'C
	and r24, r21	; r24 <- B'CD

	or r16, r24		; F0 <- (A'B + B'CD)
	com r16			; F0 <- (A'B + B'CD)'
	andi r16, 0x01	; get the lsb

	; F1 = (AC)(B + D)

	mov r24, r18	;r24 <- A
	and r24, r20	; r24 <- AC
	mov r17, r24	; F1 <- AC

	mov r24, r19	;r24 <- B
	or r24, r21		;r24 <- B + D

	and r17, r24	; F1 <- (AC)(B + D)
	andi r17, 0x01	; get the lsb
	lsl r17			; shift left

	or r16, r17
	out PORTB, r16	;show on LEDs

	rjmp main
	
\end{minted}

\subsection*{Κώδικας σε c}

\begin{minted}[linenos,frame=single,breaklines=true]{c}

#include <avr/io.h>

//Initialize variables
char F0,F1,A,B,C,D;

int main() {
	
	//Set PORTC as input
	DDRC = 0x00;
	//Set PORTB as output
	DDRB = 0xFF;
	
	while(1) {
		
		A = PINC & 0x01;		//Get PC0
		B = (PINC & 0x02) >> 1; //Get PC1 and shift to LSB
		C = (PINC & 0x04) >> 2;	//Get PC2 and shift to LSB
		D = (PINC & 0x08) >> 3;	//Get PC3 and shift to LSB
		
		//Perform bitwise operation for F0 = (A'B + B'CD)'
		F0 = ~((~A & B) | (~B & C & D));
		//Keep the LSB
		F0 = F0 & 0x01;
		
		//Perform bitwise operation for F1 = (AC)(B + D)
		F1 = (A & C) & (B | D);
		//Keep the LSB
		F1 = F1 & 0x01;
		//Shift one to the left
		F1 = F1 << 1;
		//Merge F0-F1 and output
		PORTB = F0 | F1;
		
	}
	
	
}
	
\end{minted}

\section*{Άσκηση 2}

\subsection*{Γενική ιδέα}
Τρέχουμε το πρόγραμμα μετρητή και μόλις γίνει διακοπή INT1 το πρόγραμμα πηγαίνει στην ρουτίνα εξυπηρέτησης της διακοπής, όπου ελέγχει αν τα dip switches A7 και Α6 είναι ON, τότε αυξάνει τον καταχωρητή που μετράει τις διακοπές και βγάζει το αποτέλεσμά του στην έξοδο.

\subsection*{Κώδικας}

\begin{minted}[linenos,frame=single,breaklines=true]{nasm}

.org 0x0	;η αρχή του κώδικα reset
rjmp reset
.org 0x4	;η εξυπηρέτηση της ΙΝΤ1
rjmp ISR1

reset:
	ldi r24,(1<<ISC11)|(1<<ISC10)
	out MCUCR,r24	;η διακοπή ΙΝΤ1 να προκαλείται με σήμα θετικής ακμής
	ldi r24,(1<<INT1)
	out GICR,r24	;ενεργοποίηση διακοπής ΙΝΤ1
	sei
	ldi r24,low(RAMEND)
	out SPL,r24
	ldi r24,high(RAMEND)
	out SPH,r24


;counting programm given
start:
    clr r26
	out DDRA,r26	;Αρχικοποίηση της PORTA για είσοδο
	ser r26
	out DDRC,r26	;Αρχικοποίηση της PORTC για έξοδο
	out PORTA,r26	;ενεργοποίηση αντιστάσεων πρόσδεσης
	clr r16
	clr r26

loop:
	out PORTC,r26	;δείξε την τιμή του μετρητή στην θύρα εξόδου των LED
	inc r26
	rjmp loop

ISR1:
	push r26	;σώζει το περιεχόμενο του r26
	in r26,SREG	;σώζει το περιεχόμενο των sreg
	push r26
	ser r26
	out DDRB,r26	;η θύρα Β ως έξοδος
	in r26,PINA		;διαβάζει την Α
	subi r26,192
	brlo noincr	;αν δεν είναι 1 τα PA7 και PA6 πήγαινε στο noincr
	inc r16			;αλλιώς αύξησε τον μετρητή
	out PORTB,r16	;και βγάλτον στην έξοδο
noincr:
	pop r26		
	out SREG,r26	;επανάφερε την τιμη των sreg 
	pop r26			;και του r26
	reti			; και επίστρεψε

\end{minted}

\section*{Άσκηση 3}

\subsection*{Γενική Ιδέα}

Το πρόγραμμα είναι στο άπειρο βρόχο και περιμένει διακοπή από τον χρήστη. Μόλις ανιχνεύσει διακοπή πηγαίνει στην ρουτίνα εξυπηρέτησης της διακοπής και αφού ελέγξει αν το PA2 είναι ON τότε μετράει πόσα switches είναι ανοιχτά στην Θύρα B και ανάβει στην έξοδο τόσους διακόπτες σύμφωνα με την εξίσωση $n = 2^{count} - 1$. Αν το PA2 είναι ΟFF τότε στην έξοδο βγάζει τον δυαδικό αριθμό των ανοιχτών switches της θύρας B.

\subsection*{Κώδικας}

\begin{minted}[linenos,frame=single,breaklines=true]{c}


#include <avr/io.h>
#include <avr/interrupt.h>

volatile int check, count, v;
volatile int bit_count(volatile int v);


#define NOP(){__asm__ __volatile__("nop");} //does nothing. needed inside while(1) compiler optimizations skip it

// INT0 interrupt service routine
ISR(INT0_vect) {
	check = PINA & 0x04;	//get PA2
	if (check == 0x00) {	//if not set
		v = PINB;
		count = bit_count(v); //count the number of set bits
		PORTC = (1 << count) - 1; //light up 'count' # of LEDs
	}
	else {
		v = PINB;
		count = bit_count(v);
		PORTC = count;	//output the number of set bits
	}
	return;
}

// function that counts the number of set bits in a byte
volatile int bit_count(volatile int v) {
	volatile int count = 0x00;
	int hex_lsb;
	int loop_count = 7;
	while(loop_count >= 0) {
		hex_lsb = v & 0x01;
		if (hex_lsb == 0x01) {
			count++;
		}
		v = v >> 1;
		loop_count--;
	}
	return count;
}


int main(void) {
	
	DDRA, DDRB, DDRD = 0x00; // make PORTA, PORTB, PORTD as inputs
	DDRC = 0xff;			// make PORTC as output
	//disable global interrupts
	cli();
	
	GICR = 0x40;	//enable INT0
	MCUCR = 0x03;	// enable on rising edge
	//enable global interrupts
	sei();
	
	while(1) {NOP();}
}

\end{minted}

Επίσης, σημειώνεται ότι στην εξέταση της παραπάνω άσκησης δεν μας λειτουργούσε η συνάρτηση ύψωσης σε δύναμη. Όμως, με την παραπάνω τροποποίηση στην γραμμή 16, το πρόγραμμα λειτουργεί σωστά και εκτυπώνει το επιθυμητό αποτέλεσμα. 

\end{document}
